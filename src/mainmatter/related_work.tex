\section{Related work}

There is not much work that relates directly concurrency with game theory,
and admittedly it is probably because they don't fit very well: in a concurrent
setting there is an authority that tells every player exactly what to do,
there are no rogue players and the scheduler seeks for the best possible global
outcome. Most of the research is focused on the distributed setting, where
there are independent agents that need to communicate in order to reach a 
goal, which might or might not be a common goal.
Nonetheless we think that using game theoretical concepts to shape a scheduler
would be a useful exercise.

The problem of dividing several jobs among several machines in a way 
that optimizes some global objective function is well known and has 
been widely studied in computer science. 

Premi et al \cite{9244046} modelled the resource allocation 
problem for heterogeneous platforms as a congestion game. 
The proposed cost function was dependent on the processor frequency, 
energy-per-operation of the processor, and average power consumption. 
Their goal was to find a Nash Equilibrium, which guarantees that no player 
can unilaterally improve its position. 
An experimental evaluation showed that the game worsens a little the performance, 
but it considerably reduces the power consumption, 
especially when the number of tasks is larger than the number of resources. 

Most of the related work focused on the distributed and cloud computing scenarios. 
For example, both Dong et al \cite{10.1145/2639108.2639128} 
and Yang et al \cite{7425237} dealt with energy metrics, 
trying to find a strategy that minimizes the total energy consumption of the system. 
While \cite{10.1145/2639108.2639128} used cooperative game to model the scenario, 
in \cite{7425237} a non-cooperative game was used to model a similar setting.

Song et al \cite{10.1007/978-1-4614-7010-6_33} used game theory to model MapReduce programming model, 
which is a distributed programming model designed by Google for processing large 
scale data sets in parallel. The task is somewhat aligned with our objective as well, 
since they need to select the most suitable job to be sent to the cluster given the 
competitive relationship among the jobs. They consider two games to model a two-level 
scheduling problem: auction game and task scheduling game. In an auction game, 
every job makes a bid and a job with the highest bid wins the game. 
For the second game, they use Hungarian Method to solve task assignment problem.

In order to build the job-scheduling game we followed the suggestions in paper \cite{rosenthal_class_1973}
and paper \cite{10.1145/1807406.1807411}. The former introduces and properly formalizes the class of congestion
games, while the latter formalizes repeated congestion games, and proposes strategies
to reach a Pareto optimal equilibrium with such types of games.
