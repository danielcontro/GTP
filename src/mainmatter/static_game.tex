\newpage
\section{Static game of complete information}

The first scenario considered for the scheduling problem is the one with a
single core system with no pre-emption. In this case the scheduling is
performed only for the next time slot and there are two threads, each one of
which is ready to be executed and terminated during it. The possible actions
of each thread are to either be executed (E) or not executed (N); the
outcome of a game depends on the combination of actions chosen by the players,
which on the other hand try to maximize their own payoff, built in such a way
that promotes coordination in order to reach a Pareto efficient solution.

The game associated with the given scenario is defined as follows:

\begin{itemize}

  \item $\mathcal{P} = \{T_1, T_2\}$ denotes the set of players which contains
    the two threads;

  \item For $i \in |\mathcal{P}|,\ \mathcal{S}_i$ denotes the set of strategies
    of player $i$, where each $s \in \mathcal{S}_i$ is a non-empty subset of
    the set of actions $\mathcal{A} = \{ N, E \}$ (Execute or Not execute the
    thread);

  \item the payoff functions
    $u_i(s_1, s_2): \mathcal{A} \times \mathcal{A} \to \mathbb{R}$
    for $i \in |\mathcal{P}|$ are defined using the following parametes for which are
    given different ordinal relations in order to study how the game behaves
    in the different cases: 

    \begin{itemize}
      \item $u_1(N, N) = n_1$: $T_1$'s payoff if none of the thread is executed
      \item $u_2(N, N) = n_2$: $T_2$'s payoff if none of the thread is executed

      \item $u_1(E, N) = e_1$: $T_1$'s payoff if it's the only thread to be executed
      \item $u_2(N, E) = e_2$: $T_2$'s payoff if it's the only thread to be executed

      \item $u_1(N, E) = e_2 + n_1$: $T_1$'s payoff if only $T_2$ is executed
      \item $u_2(E, N) = e_1 + n_2$: $T_2$'s payoff if only $T_1$ is executed

      \item $u_1(E, E) = c_1$: $T_1$'s payoff if both of the threads are
        scheduled for execution in the next time slot (situation that can't
        practically happen since there's no pre-emption, which in reality
        "collapses" into execute either $T_1$ or $T_2$ with a reduced payoff)
      \item $u_2(E, E) = c_2$: $T_2$'s payoff if both of the threads are
        scheduled for execution in the next time slot
    \end{itemize}
\end{itemize}

Thus the game can generally be represented with the following bi-matrix:

\begin{center}
\begin{game}{2}{2}
  $T_1 \backslash T_2$ & N & E \\
  N         & $n_1, n_2$ & $e_2+n_1, e_2$ \\
  E         & $e_1, e_1+n_2$ & $c_1, c_2$
\end{game}
\end{center}

The general ordering considered between the payoffs is the following
\[
n_1 < n_2 <= 0 < c_1 < c_2 < e_1 < e_2
\]
in which executing $T_2$ yields a payoff greater than executing $T_1$ (the
opposite would be analogous).

Since there's no pre-emption, $e_j - n_i > c_i$ (letting the other thread execute is
better than scheduling both threads for execution since the result is the same
with a reduced payoff in the latter case) there are two Pareto efficient
strategies: $(N, E)$ and $(E,N)$ which are also the two Nash Equilibriums in
pure strategies, while in mixed strategies if $\alpha$ is the probability of
$T_1$ to play $N$ and $\beta$ the probability of $T_2$ to play $N$ then
$(\alpha, \beta)$ is a NE with $$\alpha = \frac{e_1+n_2-c_2}{e_1+e_2-c_2}$$
and $$\beta = \frac{e_2+n_1-c_1}{e_1+e_2-c_1}$$

\subsection{Equality between payoffs of thread execution}

A special case of ordering between the payoffs considered is the following:
\[
n_1 = n_2 <= 0 < c_1 = c_2 < e_1 = e_2
\]
where executing either $T_1$ or $T_2$ yields the same payoff.

In this case the game can be simplified obtaining the following:

\begin{center}
\begin{game}{2}{2}
  $T_1 \backslash T_2$ & N & E \\
  N         & $n, n$ & $e+n, e$ \\
  E         & $e, e+n$ & $c, c$
\end{game}
\end{center}

Analyzing the game three Nash Equilibriums can be found: $(E, N)$
and $(N, E)$ in pure strategies (as in the previous case), while in mixed
strategies if $\alpha$ is the probability of $T_1$ to play $N$ and $\beta$
the probability of $T_2$ to play $N$ then $(\alpha, \beta)$ is a NE with
$$\alpha = \beta = \frac{e+n-c}{2e-c}$$

\subsection{Adding pre-emption}

In case that the CPU supports pre-emption it could be the case that
$c_i > e_j + n_i$, which would mean that executing part of one thread, perform
a context switch and execute part of the other yields a higher payoff than
just executing one of the threads. If this were the case then also $(E, E)$
would be a Pareto efficient solution with $(E, N)$ and $(N, E)$, moreover it
would be the only Nash Equilibrium in the game (in mixed strategies there
would be two negatives probabilities corresponding to the pure strategy $(E,E)$).
If on the other hand $c_i < e_j + n_i$ then what has previously been discussed
holds still.

From which it can be built a parametrized potential function $\Omega$:

\begin{center}
\begin{game}{2}{2}
  $T_1 \backslash T_2$ & N & E \\
  N         & $0$ & $e_2-n_2$ \\
  E         & $e_1-n_1$ & $c_1-(e_1-n_1)$
\end{game}
\end{center}

since in general $n_1,n_2 <=0 < c_1, c_2 < e_1, e_2$ and $c - (e-n) < 0$,
because there's no pre-emption so scheduling concurrent execution rather than
executing only one thread is disadvantageous, it follows that $\Omega$ has the
following positivity:

\begin{center}
\begin{game}{2}{2}
  $T_1/T_2$ & N & E \\
  N         & $0$ & $>0$ \\
  E         & $>0$ & $<0$
\end{game}
\end{center}