\section{Introduction}
A computer system may be organized in a number of different ways, 
one of which can be categorized according to the number of 
general-purpose processors used. On a single-processor system, 
there is one main CPU capable of executing a general-purpose instruction 
set, including instructions from user processes.
Multiprocessor systems, on the other hand, have two or more processors 
in close communication, sharing the computer bus and sometimes the clock
memory, and peripheral devices \cite{10.5555/2490781}.

The multiprocessor systems can use asymmetric multiprocessing, 
where processors have a master-slave relationship, 
or symmetric multiprocessing, in which each processor performs all 
operations within an operating system. 
The benefit of this model is that many processes can run simultaneously 
without causing a significant deterioration of performance. 
However, since the CPUs are separate, one may be sitting idle while 
another is overloaded, resulting in inefficiencies \cite{10.1145/1463822.1463838}.

A common practice today is to include multiple computing cores on a 
single chip. 
Those multiprocessor chips can be more efficient due to fast 
on-chip communication, and they consume significantly less power than 
multiple single-core processors. 
In our paper, we will consider two games: with one single-core processor 
and with one multi-core processor. 

Time sharing and multiprogramming require that several jobs be kept simultaneously in memory. 
If several jobs are ready to be brought into memory, and if there is not enough room for all of them,
then the system must choose among them. 
Making this decision is called job-scheduling. 

Our proposal is to adopt game theory as a way to tackle the job-scheduling process, 
both for single-core and multi-core processors. 
Game theory is a branch of mathematics that studies strategic decision-making. 
In the context of resource allocation, game theory can be used to analyse how 
different players in a system might make decisions about how to allocate resources, 
given their own goals and constraints, and the actions and objectives of other players. 

When applying game theory to job scheduling, jobs are identified as rational players, 
which makes sense given the competitive nature of multi-programming. 
Actually, we can consider job scheduling as a dynamic game, where the utility 
function for the job can be calculated based on the time it took from the job being 
created to being executed. However, to allow a for a more fine-grained study we will 
focus on static games with complete information.

One common application of game theory in resource allocation is the study of auctions, 
in which multiple players bid for a limited resource. Auctions can be analysed using 
game theoretical concepts such as equilibrium, efficiency, and fairness \cite{9244046}. 
An important step is to come up with an efficient strategy, so that the pull of jobs
will not suffer from a condition called starvation, when some jobs in the pull will
never be executed because of the incorrect scheduling.

In game theory, there exists a job scheduling game, which is a game that models a 
scenario when multiple selfish users wish to utilize multiple processing machines. 
Each user has a single job, and he needs to choose a single machine to process it. 
The incentive of each user is to have his job run as fast as possible \cite{noauthor_job_2021}. 
This game can be used as a base for our model, and we will explore the job-scheduling 
in several iterations, where each player needs multiple rounds in order to complete its task.