@book{10.5555/2490781,
author = {Silberschatz, Abraham and Galvin, Peter B. and Gagne, Greg},
title = {Operating System Concepts},
year = {2012},
isbn = {1118063333},
publisher = {Wiley Publishing},
edition = {9th},
abstract = {The ninth edition of Operating System Concepts continues to evolve to provide a solid theoretical foundation for understanding operating systems. This edition has been updated with more extensive coverage of the most current topics and applications, improved conceptual coverage and additional content to bridge the gap between concepts and actual implementations. A new design allows for easier navigation and enhances reader motivation. Additional end-of-chapter, exercises, review questions, and programming exercises help to further reinforce important concepts. WileyPLUS, including a test bank, self-check exercises, and a student solutions manual, is also part of the comprehensive support package.}
}

@inproceedings{10.1145/1463822.1463838,
author = {Conway, Melvin E.},
title = {A Multiprocessor System Design},
year = {1963},
isbn = {9781450378833},
publisher = {Association for Computing Machinery},
address = {New York, NY, USA},
url = {https://doi.org/10.1145/1463822.1463838},
doi = {10.1145/1463822.1463838},
abstract = {Parallel processing is not so mysterious a concept as the dearth of algorithms which explicitly use it might suggest. As a rule of thumb, if N processes are performed and the outcome is independent of the order in which their steps are executed, provided that within each process the order of steps is preserved, then any or all of the processes can be performed simultaneously, if conflicts arising from multiple access to common storage can be resolved. All the elements of a matrix sum may be evaluated in parallel. The ith summand of all elements of a matrix product may be computed simultaneously. In an internal merge sort all strings in any pass may be created at the same time. All the coroutines of a separable program may be run concurrently.},
booktitle = {Proceedings of the November 12-14, 1963, Fall Joint Computer Conference},
pages = {139–146},
numpages = {8},
location = {Las Vegas, Nevada},
series = {AFIPS '63 (Fall)}
}

@online{noauthor_job_2021,
	title = {Job scheduling game},
	copyright = {Creative Commons Attribution-ShareAlike License},
	url = {https://en.wikipedia.org/w/index.php?title=Job_scheduling_game&oldid=1004213597},
	abstract = {In game theory, a job scheduling game is a game that models a scenario in which multiple selfish users wish to utilize multiple processing machines. Each user has a single job, and he needs to choose a single machine to process it. The incentive of each user is to have his job run as fast as possible.},
	language = {en},
	urldate = {2023-01-20},
	journal = {Wikipedia},
	month = feb,
	year = {2021},
	note = {Page Version ID: 1004213597},
}


@inproceedings{9244046,
author={Premi, Lara and Reghenzani, Federico and Massari, Giuseppe and Fornaciari, William},
booktitle={2020 International Conference on Embedded Software (EMSOFT)}, 
title={A Game Theory Approach to Heterogeneous Resource Management: Work-in-Progress}, 
year={2020},
volume={},
number={},
pages={25-27},
doi={10.1109/EMSOFT51651.2020.9244046}
}
