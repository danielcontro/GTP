[EDIT REF 5] @inproceedings{10.1145/2639108.2639128,
author = {Dong, Mian and Lan, Tian and Zhong, Lin},
title = {Rethink Energy Accounting with Cooperative Game Theory},
year = {2014},
isbn = {9781450327831},
publisher = {Association for Computing Machinery},
address = {New York, NY, USA},
url = {https://doi.org/10.1145/2639108.2639128},
doi = {10.1145/2639108.2639128},
abstract = {Energy accounting determines how much a software principal contributes to the total system energy consumption. It is the foundation for evaluating software and for operating system based energy management. While various energy accounting policies have been tried, there is no known way to evaluate them directly simply because it is hard to track all hardware usage by software in a heterogeneous multicore system like modern smartphones and tablets.In this work, we argue that energy accounting should be formulated as a cooperative game and that the Shapley value provides the ultimate ground truth for energy accounting policies. We reveal the important flaws of existing energy accounting policies based on the Shapley value theory and provide Shapley value-based energy accounting, a practical approximation of the Shapley value, for battery-powered mobile systems. We evaluate this approximation against existing energy accounting policies in two ways: (i) how well they identify the top energy consuming applications, and (ii) how effective they are in system energy management. Using a prototype based on Texas Instruments Pandaboard and smartphone workload, we experimentally demonstrate existing energy accounting policies can deviate by 400% in attributing energy consumption to running applications and can be up to 25% less effective in system energy management when compared to Shapley value-based energy accounting.},
booktitle = {Proceedings of the 20th Annual International Conference on Mobile Computing and Networking},
pages = {531–542},
numpages = {12},
keywords = {energy management, mobile systems, energy accounting},
location = {Maui, Hawaii, USA},
series = {MobiCom '14}
}

[EDIT REF 6] B. Yang, Z. Li, S. Chen, T. Wang, and K. Li. Stackelberg game approach
for energy-aware resource allocation in data centers. IEEE Transactions
on Parallel and Distributed systems, 27(12):3646–3658, 2016.


[EDIT REF 7] @InProceedings{10.1007/978-1-4614-7010-6_33,
author="Song, Ge
and Yu, Lei
and Meng, Zide
and Lin, Xuelian",
editor="Wong, W. Eric
and Ma, Tinghuai",
title="A Game Theory Based MapReduce Scheduling Algorithm",
booktitle="Emerging Technologies for Information Systems, Computing, and Management",
year="2013",
publisher="Springer New York",
address="New York, NY",
pages="287--296",
abstract="A Hadoop MapReduce cluster is an environment where multi-users, multi-jobs and multi-tasks share the same physical resources. Because of the competitive relationship among the jobs, we need to select the most suitable job to be sent to the cluster. In this paper we consider this problem as a two-level scheduling problem based on a detailed cost model. Then we abstract these scheduling problems into two games. And we solve these games in using some methods of game theory to achieve the solution. Our strategy improves the utilization efficiency of each type of the resources. And it can also avoid the unnecessary transmission of data.",
isbn="978-1-4614-7010-6"
}
