\section{Related work}

There is not much work that relates directly concurrency with game theory,
and admittedly it is probably because they don't fit very well: in a concurrent
setting there is an authority that tells every player exactly what to do,
there are no rogue players and the scheduler seeks for the best possible global
outcome. Most of the research is focused on the distributed setting, where
there are independent agents that need to communicate in order to reach a 
goal, which might or might not be a common goal.
Nonetheless we think that using game theoretical concepts to shape a scheduler
would be a useful exercise. 

In order to build the job-scheduling game we followed the suggestions in paper \cite{rosenthal_class_1973}
and paper \cite{10.1145/1807406.1807411}. The former first introduces and properly formalizes the class of congestion
games, while the second formalizes repeated congestion games, and proposes strategies
to reach a Pareto optimal equilibrium with such types of games.
